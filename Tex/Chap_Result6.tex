
\chapter{结论和展望}
\label{chap:result6}

\section{本论文总结}

随着超算领域步入E级计算的时代,单CPU的多核化、众核化的趋势越来越明显,多线程、高并发、大规模是高性能计算的常态,并行计算已经不是超算领域所独享的技术,分布式并行编程逐渐成为应用软件,数值计算库乃至工具开发的基本编程方式,开发性能优越的软件的门槛逐渐提高,甚至具有相当的挑战性。为了缩短科学应用软件在不同硬件之间的移植周期,减少或消除应用原型程序移植到并行计算环境的时间,最大限度的提升科学研究的效率,需要更加高效、便捷的并行编程工具帮助开发人员进行并行优化,使开发人员在获得性能提升的同时,能将主要精力放在程序的算法上,而不是过多的关注并行的细节问题。

但是目前的并行计算领域还没有成熟的并行优化标准解决方案,主流的并行编程工具(如OpenMP、MPI)一般都具有二三十年的历史,采用的编程模型也是传统的BSP模型,不仅编程和调试复杂,计算资源的利用率和程序的可扩展性也不理想。针对这种现状,我们针对计算资源的利用率和可扩展性两方面的问题,提出了在共享内存和分布式内存下的并行优化解决方案,采用数据流计算模型,设计高效的任务调度策略,同时提供简单易用的用户API。

第三章介绍了共享内存环境下基于线程池和数据流模型的star\_tp的设计和开发。线程池的使用在很大程度上减少了大量线程的创建和销毁带来的系统开销;worker对DAG更新的工作量的分担缓解了master/salve架构下调度器的单点负担,防止调度器过早出现性能瓶颈;数据流模型的采用使得任何时刻满足依赖的任务都可以异步的并行执行,这使CPU空转时间大大减少,提升了CPU的利用效率。

第四章介绍了分布式下环境下基于PGAS模型和数据流模型的star-d的设计和开发。PGAS模型封装了大部分底层通信的细节问题,让开发人员像访问共享内存一样访问分布式内存,这极大地方便了star-d的分布式架构的设计;存储任务和worker状态的分布式数据结构使用单向通信进行访问,减少了双向通信造成的大量阻塞,使任务调度更加迅速,调度器可以及时发现就绪任务并将就绪任务分发到worker上;通过增加节点数,star-d可以达到的并行度也线性增长,表现出了良好的可扩展性。

通过第五章对InSAR算法的并行优化的应用实践,star\_tp和star-d都表现出不错的优化效果,不仅提升了性能,计算资源利用率也有所提升,达到了预期的研究目标。

\section{展望}

并行优化不仅在高性能计算领域很重要,在工业界也开始发挥实际作用。本论文开发的并行优化工具虽然达到了一定的优化效果,但是仔细分析的话还是有许多可以改进的空间。

star\_tp和star-d都不具有可容错性。当计算规模特别大,节点数量大量增加时,可容错性是衡量并行计算工具是否实用的重要指标。

当计算资源规模增大,节点和计算任务的差异增大时,负载均衡将对性能产生巨大的影响,star-d虽然具有一定的负载均衡能力,但是并没有考虑数据局部性、计算资源监控等复杂的负载均衡算法设计。

在架构支持上,目前只支持了共享内存和分布式内存架构,随着人工智能和深度学习的应用越来越多,异构系统越来越多,支持异构系统的并行优化工具也很有应用需求。

在star-d的实现中,虽然使用PGAS方便了分布式调度系统的实现,但是通过调研发现,目前PGAS的研究还不成熟,许多实现了PGAS的工具(如UPC++)使用时还存在许多问题,而且性能上的优势并不明显,可以借鉴高并发服务器的设计思想设计高效的分布式任务调度系统,虽然编程复杂性增加,但是可以实现更好的性能,和更高的稳定性。

综上所述,在今后的并行优化研究中在容错性、负载均衡、支持异构系统和高并发设计方面都有研究的必要性和可行性。


