\chapter{致\quad 谢}

三年的研究生生涯转眼间已经接近尾声,经过三年的学习和研究,我的毕业论文顺利的完成了。非常感谢在这期间给予我指导和关心的老师,始终陪伴我、鼓励我的亲人和朋友,还有朝夕相处、共同努力的同学们。

首先要感谢的是我的导师尤海航老师。尤老师在分布式并行计算方面具有深厚的学术造诣,在研究方向上给了我很多指导和帮助。在科研方向上,他持续关注相关领域发展方向,指导我们跟进科研的最前方;在项目开发中,在架构设计和技术方面,他给我了我很多启发和建议,让我在项目开发的过程中少走了很多弯路;在生活中,他平易近人,像对待朋友一样与我们相处,对生活上的困难他热心的给予帮助。能有尤老师这么优秀的导师是我硕士期间的幸运,在今后的工作和生活中,我会更加努力,为我们的研究小组做更多的贡献,不遗余力的回报尤老师的谆谆教诲!

其次感谢范东睿老师为我们提供了这么好的科研平台,祝愿实验室在您的带领下取得更大的辉煌。感谢张志敏老师、叶笑春老师给我的毕业论文提出的指导和建议。感谢关镇老师、师圣老师和文潇乐老师在科研、生活上的帮助和关心。感谢赵巍老师和宋敏老师在生活和毕业指导上给予的认真负责的帮助和支持。

还要感谢实验室的同学们在科研道路上的陪伴和互相帮助。感谢师兄刘欢、李亚洲和戴俊凯在并行计算方向给予我的指导,感谢师姐初一、师弟杨润楷在生活和项目开发上给予我的帮助和关心。

最后,我要感谢我的家人。感谢父母的养育之恩,感谢他们教我做人,让我有了积极乐观的性格,感谢他们对我的无私奉献和支持。感谢我的女朋友吕扬阳学,我们纯真的爱情让我的学习和生活更加充满动力!

再次感谢所有帮助我、关心我的人,是你们让我的研究生生涯如此精彩!我将怀着感恩的心准备步入新的人生阶段,愿我们的未来更加美好!


\chapter{作者简历及攻读学位期间发表的学术论文与研究成果}

\section*{作者简历:}

2011年09月-2015年06月,在北京科技大学计算机与通信工程院(系)获得学士学位。

2015年09月-2018年06月,在中国科学院计算技术研究所攻读硕士学位。

\section*{攻读硕士学位期间参加的科研项目:}

[1] 计算所课题,“STAR: Simultaneous Tuning Automatic Runtime-分布式并行运算环境的通用优化研究”,2015年09月-2016年12月。

[2] 课题组项目“中国科学院遥感与数字地球研究所INSAR系统优化”,2016年05月-2017年12月。

\section*{攻读硕士学位期间的获奖情况:}

[1] 2017年被评为中国科学院计算技术研究所“优秀学生”。

